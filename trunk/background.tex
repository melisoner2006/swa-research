\section{BACKGROUND AND LITERATURE REVIEW}

In this section, we discuss adjacent fields of research and summaries of what we
found to be important points of interest.

\begin{comment}
\subsection{Social and Professional Issues}
ACM/IEEE establish international curricular guidelines for undergradute programs
in computing in volumes CS2001, CS2008, and CS2013.  CS2013 specifies the
following principles:

\begin{enumerate}
  \item Computer Science curricula should be designed to provide students with
  the flexibility to work across many disciplines\ldots
  \item Computer Science curricula should be designed to prepare graduates for a
  variety of professions, attracting the full range of talent to the field\ldots
  \item CS2013 should provide guidance for the expected level of mastery of
  topics by graduates\ldots
  \item CS2013 must provide realistic, adoptable recommendations that provide
  guidance and flexibility, allowing curricular designs that are innovative ad
  track recent developments in the field\ldots
  \item The CS2013 guidelines must be relevant to a variety of institutions\ldots
  \item The size of the essential knowledge must be managed\ldots
  \item Computer Science curricula should be designed to prepare graduates to
  succeed in a rapidly changing field\ldots
  \item CS2013 should identify the fundamental skills and knowledge that all
  computer science graduates should posses providing the greatest flexibility in
  selecting topics\ldots
  \item CS2013 should provide the greatest flexibility in organizing topics into
  courses and curricula\ldots
  \item The deveopment and review of CS2013 must be broadly based\ldots
\end{enumerate}
\end{comment}

\subsection{Computer Literacy}

\subsubsection{Productivity Software}

\subsubsection{Computer Security}

\subsection{Computer History}

\subsection{Computers and Society}

\subsection{IT Key Qualifications}

\subsection{Course Design Process}