\documentclass[]{article}
\usepackage[margin=1in]{geometry}

\title{Article Summaries}
\author{Melis \"{O}ner, David Kaczynski}

\begin{document}

\maketitle

\section{Enhancing the Social Issues Components in our Computing Curriculum:
Computing for the Social Good 2013}

\subsection{Notes}

\begin{itemize}
  \item establishes international curricular guidelines for undergraduate
  programsin computing
\end{itemize}

\subsection{Quotes}

\begin{itemize}
  \item ``international curricular guidelines for undergraduate
  programsin computing'' (p 5, line 4)
  \item ``The development of curricular guidelines for Computer Science is
  particularly challenging given the rapid evolution and expansion of the
  field'' (p 5, line 16-17)
  \item ``The education that undergraduates in Computer Science receive must
  adequately prepare them for the workforce in a more holistic way than simply
  conveying technical facts.'' (p 11, line 155-156)
  
\end{itemize}

\section{Enhancing the Social Issues Components in our Computing Curriculum:
Computing for the Social Good 2008}

\subsection{Notes}

\begin{itemize}
  \item Social and Professional Issues (p 92)
\end{itemize}

\subsection{Quotes}

\begin{itemize}
  \item 
  
\end{itemize}

\section{What are IT Key Qualifications}

\subsection{Notes}

\begin{itemize}
  \item 
\end{itemize}

\subsection{Quotes}

\begin{itemize}
  \item The previous section showed that IT key skills have to be seen as
key skills plus IT content. Therefore it is very difficult to define
sustainable IT key skills, as the IT content is changing so rapidly.
\end{itemize}

\section{Computer Literacy: Essential in Today’s
Computer-Centric World - Gireesh K. Gupta}

\subsection{what is computer literacy ?}

\begin{itemize}
	\item Computer literacy can be defined as an individual’s ability
to operate a computer system, have basic understanding of
the operating system to save, copy, delete, open, print
documents, format a disk, use computer applications
software to perform personal or job-related tasks, use Web
browsers and search engines on the Internet to retrieve
needed information and communicate with others by
sending and receiving email. In academics, a computer
literate student should be able to apply the knowledge of
computer technology to do research and perform tasks
related to his major discipline [7].

	\item It is not enough to play games and surf the Internet to
be computer literate. You do not need to be a good
programmer or an expert in computer communications and
networking [9], nor do you need a college degree in the
computer field to be computer literate. Computer literacy
deals with being able to use the computer applications
rather than writing software [4]. A computer literate uses
the computer technology to perform his job more
effectively and efficiently.
\end{itemize}

\subsection{Data on Computers-In-Use Globally}
In this section there is numerical data about the usage of computers. But all numbers belong to the years between 2005 and 2007. It can be used or more up-to- date data can be gathered by using the references.
\subsection{Benefits of Computer Literacy}
Computer literacy can be very rewarding for people of
all ages – children, teenagers, adults, and senior citizens.
Some of the benefits are described below.
\begin{itemize}

\item finding Information in the Internet:
health, cooking,
nutrition, entertainment, weather report, road maps and
directions, etc.
\item  online shopping and online banking
\item electronic commucication : e-mailing
\item job searchong for 7/24

\end{itemize}

\subsection{Objectives of CPS 100}

1. Describe and discuss the importance of data as a
business asset.\\
2. Identify the basic parts and functions of
information systems.\\
3. Identify the devices that comprise a computer
system and describe the functions of each.\\
4. Describe the role and function of system software.\\
5. Identify and discuss some of the issues faced by
the CIS (Computer Information Systems)
profession and society at large, including the
topics of security, privacy of data, and intellectual
property rights.\\
6. Identify and be able to discuss the effects
computers are having on individuals, businesses,
schools, homes, and governmental agencies,
including databases and data communications.\\
7. Use a microcomputer system and its operating
system.\\
8. Enter text into, revise text in, and print text using a
word processor.\\
9. Use a spreadsheet to construct simple models
using formulas, to dynamically revise the model,
and to print the spreadsheet data.\\
10. Prepare a simple computer presentation.\\
11.**** Access remote computers to send and receive
information.\\

\maketitle

\section{Attitudes towards Computer
Science-Computing Experiences as a Starting Point
and Barrier to Computer Science}
In which way do computing experiences shape attitudes towards
computer science?

\subsection{CS Programs - Student Enrollment Rates in Germany and US }

It can be retrieved that some statistical information of students' low enrollment and high dropout rates in the computing classes in US and Germany from this part. It is also touching on the gender difference in the field.

\subsection{Reasons for Low Levels of Interest}

This was done mainly in gender research.
For example, male and female high school students conceptualize
CS as a male field [27], [29], putting off women [36].
\subsection{Role of Computing Experiences}

He surveyed students of a
secondary-level computing course and observed that 92.9\% possessed
and used a computer at home, but “the majority of students
(over 58 percent) were unable to approximate a definition of
computer science.” (pp. 149).\\
Beaubouef and Mason state also that “students
often have misconceptions about the field of computer science.
Many of them take a computer literacy course, do well in it, and
believe that computer science is all about word processors,
spreadsheets, or web browsers.” ([3], pp. 103).\\

Teachers should take into account prior experiences of their
students and be aware of students' preconceptions. Therefore
more research is needed to understand such experience-based
entry barriers ([8], pp. 153) and the effect of computing experiences
on attitudes towards CS, including perceptions of the field
as well as issues of self-confidence. -> That's why we should improve the lab exercises because most of the students already have experience.
\subsection{Research Approach}
We assume that\textbf{ computing experiences }can foster interest in CS
and motivation to pursue a career in this field if the experiences
are rewarding and lead to a development of CS related skills and
understanding.


\end{document}


