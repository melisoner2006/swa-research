\section{INTRODUCTION}

Central Michigan University offers a computer science course for non-majors,
CPS 100:  Computers and Society.  The course has an enrollment of several
hundred students each year, largely due to being a compulsory course in several
degree programs.  The course covers a wide range of topics, including:

%what is our class about

\begin{itemize}
  \item history of computer hardware and software,
  \item societal impacts of computer technology, 
  \item proficiency in navigating the Internet, and
  \item proficiency in Mirosoft Office and Microsoft Windows software.
\end{itemize}

CPS 100 is taught with equal parts of lecture and lab periods.  In the case of
the lab, the exercises have not been updated for several years.  This
has lead to many lab exercises being inoperable due to outdated instructions for
user interfaces for software products, such as Google and Windows.  We decided
to take this as an opportunity to overhaul the lab exercises with modern computer science objectives for non-majors.  The resulting course objectives can be described
as applying ACM / IEEE international curricular guidelines for undegraduate
programs in computing in the context of IT key qualifications as described by
D\"{o}rge and Schulte.  In other words, the goal of this course is to teach the
social and professional issues recommended by IEEE / ACM.  These topics are
presented through lab exercises that require the students to apply key
qualification skills, such as the ability to cooperate, flexibility, creativity,
autonomy, and more.

For the purposes of this research, we choose to define a process for updating
computer science curriculum for non-majors and measuring the progress of the
transition.  The process is summarized here:

\begin{enumerate}
  \item Collect available metrics, such as grade distributions, completion
  rates, and attitude surveys.
  \item Define the scope of the course (history, ethics, software proficiency,
  etc.).
  \item Define two sets of course objectives:  key qualifications, and social
  or professional issues.  Define criteria for successful mastery of
  course objectives.
  \item For all existing lab exercises, attempt to create an optimal
  relationship to course objectives.  An optimal relationship is defined by the
  following two attributes:
  \begin{itemize}
    \item A lab exercise addresses exactly one key qualification and one
    social or professional issue.
    \item Each key qualification and social or professional issue is addressed
    by no more than one exercise.
  \end{itemize}
  \item Create lab exercises for the remaining course objectives, attempting to
  create optimal relationships when possible.
  \item Implement updated lab exercises.
  \item Collect available metrics and analyze the results.
  \item Repeat the process, refining the scope, course objectives, and mastery
  criteria as needed.
\end{enumerate}

The following sections include details of key qualifications and social
or professional issues, the benefits of mastering such course objectives, and
methods for applying the previousy defined process.

%what is the class about in other universities

%define computer literacy

%why is computer literacy important

	%consumer-awareness 
	
	%computer waste?
	
	%computer security for Internet users
	
	%future policy makers?

%introduce method/process for changing class